\chapter{Technical Proposal for an IPOL Matlab Demo Server}


\section{Solution from Juan}:

There's a matlab code submitted by author, with possibility to add some exotic/personnal toolbox

\subsection{Compilation}
\begin{itemize}
\item a wrapper CPP source file that serves as entry point (main), handles input data and parameters and passes them to the matlab code of the author.
\item the server compile the binaries by using a matlab script that compile with "mcc" the matlab code and with "mbuild" construct the execution binaries which NEED a matlab key.
\end{itemize}


\subsection{Execution}
The generated binary depends on Matlab runtime libraries.\\

So it needs :
\begin{itemize}
\item the sources
\item the cpp wrapper (can be generated)
\item the script (can be generate) 
\item a Matlab Compiler Runtime Environment (MCR) that need nothing special to run
\item a person that can compile WITH a matlab token for compilation with the same version AS the MCR. This happens only one time, offline and can wait for license token.
\end{itemize}

Note: at the end for running a submission, it needs only one key to compile and none to run.

\section{Server modifications}

The server needs to have MCR installed no further modifications are required.

\section{Distribution of workload}
\subsection{Author job}

\begin{itemize}
\item Put his sources into a way that it is easy to use
\begin{itemize}
\item a scr folder
\item  a toolbox folder (with all the .m files that is needed for execution but is not the algorithm)
\end{itemize}
\item They will be probably required to comply with a certain matlab version.
\item They will have to comply with a certain input/ouput interface in their matlab main function.
\end{itemize}



\subsection{Editor job (matlab team)}

- generate or write the cpp file AND the script 
- compile with the same version as the MCR the code for generating the binaries

\subsection{Reviewer job}
-The scientific review stays unchanged
-The software review

\section{Update policy}
Any given matlab usually survives quite well for many years (3 or 4 server OS(linux) versions).


\section{Problems to Solve}

\begin{itemize}
\item how do we update the MCR ?
\item if it not update :
\begin{itemize}
\item no author in the future
\item One solution is to update and if a demo doesn't work anymore let it die, or put in a queue for update.
\end{itemize}
\item if it updates :  will the old demo work?
\item Licensing do we need to RE-compile every programs everytime we update the MCR ? \\
Is it possible to run concurrent matlab processes without a license token?
Yes, using the MCR (Matlab Compiler Runtime) running the demos is free and legal. Tokens are needed only to compile (one token 1 time).
\end{itemize}

\section{How it will works (submission)}

Steps : 
\begin{enumerate}
\item We receive a matlab code with article 
\item An editor analyses the code according to the software guidelines (to be defined)
\item The editor writes (or automatically generates) the wrapper cpp file, compiles and generates the final binary
\item The editor or author writes a demo calling the generated binary
\item The rest of the process is the same as a regular IPOL publication
\end{enumerate}



\section{Software Guidelines}

This has to be defined, but some hints are provided:
\begin{itemize}
\item Matlab users are used to plot data interleaved in the code. Authors should be required to use a flag to turn on/off the plots. It is advised not to remove them, because it might be useful for review.
\item The input/output interface of the main matlab function has to comply with a certain structure.
\end{itemize}

