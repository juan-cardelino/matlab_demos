%-------------------------------------------------------------------------------
\section{Compilation How To}\label{sec:compilation}
%-------------------------------------------------------------------------------
\subsection{Introduction}
MATLAB programs can be compiled into standalone executable applications using
the MATLAB compiler.  The MATLAB compiler can be invoked from within MATLAB or
from the command line. For example, in the \emph{purple} server the location of the
MATLAB compiler is \texttt{/usr/local/MATLAB/bin/mcc}.

%-------------------------------------------------------------------------------
\subsection{Compiling a simple script}
Compiling a simple MATLAB script is very easy. Suppose we have created the
following simple script, and saved it as \texttt{test\_script.m}
\begin{lstlisting}[numbers=left,title={First version of \texttt{test\_script.m}},captionpos=b]
function test_script()

a = 1;

disp('Hello World.');
\end{lstlisting}

Compiling it from within MATLAB would require executing the following command:

\begin{lstlisting}
mcc -m test_script
\end{lstlisting}

Compiling it from outside MATLAB (from command line) would require calling the
\texttt{mcc} script. In \emph{purple} this would be done as:

\begin{lstlisting}[language=bash]
/usr/local/MATLAB/bin/mcc -m test_script
\end{lstlisting}


Once the MATLAB compiler has finished, \textbf{it will produce two important
  files}: One is the executable binary file, that in this case would be called
\texttt{test\_script}. The other one is a bash script, which in this case would
be called \texttt{run\_test\_script.sh}. The bash script sets the enviornment
variables needed to load the necessary libraries for running the script. It will
be this bash script that will need to be called from the IPOL demo.  The bash
script takes as first argument the path to the MATLAB root directory
(\texttt{/usr/local/MATLAB} in \emph{purple}). The rest of the arguments are passed to
the MATLAB script. Our simple script does not require any argument, so if we
were in \emph{purple}, it would be simply called by:
\begin{lstlisting}[language=bash]
./run_test_script.sh /usr/local/MATLAB
\end{lstlisting}


%-------------------------------------------------------------------------------
\subsection{Passing arguments to the script}
So far our example script was very simple, but a useful script will probably
require passing arguments to it. Fortunately this is also very simple, since
arguments passed to the bash script are directly passed to the compiled MATLAB
script. The only important thing to have in mind is that \textbf{a compiled
  MATLAB script receives the arguments as character strings}. See the following
snippet of code for a new version of our test script.

\begin{lstlisting}[numbers=left,title={Second version of \texttt{test\_script.m}},captionpos=b]
function y = test_script(x)

if isdeployed
    x = str2num(x);
end

y = 2 * x;

disp(sprintf('Your result is %f',y));
\end{lstlisting}

Let us examine lines 3 to 5. The condition tested in line 3 uses the
\texttt{isdeployed} MATLAB function, which yields a \texttt{true} value when the
script is run a compiled code, and yields \texttt{false} when it is run as a
normal \texttt{.m} script inside MATLAB. Since compiled MATLAB code receives the
arguments as strings of characters, the sentence in line 4 converts the argument
to a numeric variable.

After compilation, an invocation to this script in \emph{purple}, passing the number 3.5
as argument would be done as:

\begin{lstlisting}[language=bash]
./run_test_script.sh /usr/local/MATLAB 3.5
\end{lstlisting}

And the output should be:

\begin{lstlisting}[language=bash]
Your result is 7.000000
\end{lstlisting}

Note that other useful types of arguments are matrices or filenames. Matrices
need to be specified inside double quotes. The following example call would pass as arguments the filename ``lena.png'' and the matrix $[3,~2,~1]$:

\begin{lstlisting}[language=bash]
./run_test_script.sh /usr/local/MATLAB lena.png "[3 2 1]" 
\end{lstlisting}

%-------------------------------------------------------------------------------
\subsection{Using multiple \texttt{.m} files}

A normal MATLAB project will proably consist of multiple \texttt{.m} files, and
a main file for executing the program. This needs to be communicated to the
MATLAB compiler \texttt{mcc} using the \texttt{-a} option.  For example, if our
script \texttt{test\_script.m} calls other \texttt{.m} scripts located in a
local folder called \texttt{toolbox}, the compiler invocation would need to look like
this:

\begin{lstlisting}[language=bash]
/usr/local/MATLAB/bin/mcc -a ./toolbox -m test_script
\end{lstlisting}

This is telling the compiler to add all files in the \texttt{toolbox} folder and
all subfolders. Note that the behaviour of compiled code is different than when
running a normal \texttt{.m} script inside MATLAB. If we were running the script
inside MATLAB, we would probably use the command \mcode{addpath ./toolbox} to
tell MATLAB to look for \texttt{.m} files inside the \texttt{toolbox}
folder. This is not possible in compiled code. In fact using the
\texttt{addpath} function inside a script that will be compiled will probably
produce an error at runtime causing the termination of the program. The
workaround for this is once again to use the \texttt{isdeployed} function to
check if the script is run as compiled code or as a regular MATLAB script.

Suppose we create a folder called \texttt{toolbox} and inside it a file named \texttt{multiply.m} containing:

\begin{lstlisting}[numbers=left,title={\texttt{multiply.m}},captionpos=b]
function c = multiply(a,b)

c = a * b;
\end{lstlisting}

The new version of our \texttt{test\_script.m} would look like this:

\begin{lstlisting}[numbers=left,title={Third version of \texttt{test\_script.m}},captionpos=b]
function y = test_script(x)

if isdeployed
    x = str2num(x);
else
    addpath ./toolbox
end

y = multiply(2,x);

disp(sprintf('Your result is %f',y));
\end{lstlisting}

And the compilation could be done as any of the following:

\begin{lstlisting}[language=bash]
/usr/local/MATLAB/bin/mcc -a ./toolbox -m test_script
\end{lstlisting}

\begin{lstlisting}[language=bash]
/usr/local/MATLAB/bin/mcc -a ./toolbox/*.m -m test_script
\end{lstlisting}

\begin{lstlisting}[language=bash]
/usr/local/MATLAB/bin/mcc -a ./toolbox/multiply.m -m test_script
\end{lstlisting}

%-------------------------------------------------------------------------------
\subsection{Displaying figures}
If your MATLAB code uses the \texttt{figure()} function to display plots, it is
possible that the compiled program when executed, waits for the closing of the
figures windows to terminate. Since in \emph{purple} you do not have access to the
graphical interface, we recommend turning off the displaying of figures in the
compiled program, by adding the following option to the compilation command
\texttt{mcc}:
\begin{lstlisting}[language=bash]
-R -nodisplay
\end{lstlisting}
The \texttt{-R} option tells the compiler the MATLAB options that should be used
at runtime.
