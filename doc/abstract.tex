
In the context of increasing the number of authors publishing on IPOL, it was proposed to reduce the amount of work required by the authors to submit an article for IPOL.
Two of the main sources of extra work identified are the level of exigence of the software guidelines for the current language supported (C/C++), and the possibility of including new languages and the particular case of Matlab.
In this report we propose an analysis of the problem and a roadmap to implement Matlab support on IPOL. Only the main ideas is discussed here and the technical discussion is addressed in the extended report.

Matlab is a widely used tool in our research community and many researchers students only know how to work in Matlab. Transcribing this code to C/C++ is perceived as a huge barrier to publish in IPOL. This is specially relevant in the Audio Processing Community, where almost everyone codes in Matlab.

From the preliminary test performed, and further discussion with the team, we conclude that it is possible to have Matlab demos using the same server, with a very small increase in the workload of IPOL staff. This extra workload will be handled by and external team. More testing is needed but the main technical problems were tested.
In addition, the team should design new Software Guidelines and propose a slightly modified submission process.
The main goal is to have some matlab publications in a one year period.

